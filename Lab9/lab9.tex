\documentclass[12pt]{article}
\usepackage{adjustbox}
\usepackage{fixltx2e}
\usepackage{amssymb,amsmath,epsfig,rawfonts}
\usepackage{theorem,latexsym}
\usepackage{amsmath,amssymb,natbib}
\usepackage{multicol,multirow,epsfig}
\usepackage{makeidx}  %% use this pack to add an index page
\makeindex   
\usepackage[a4paper, total={7in, 10in}]{geometry}
\title{CENG 391 Introduction to Image Understanding}
\date{December 22, 2017}

\begin{document}
	\maketitle
	\begin{center}\textbf{Homography Estimation with RANSAC}\end{center}
Write a C++/Python program that  operates the following tasks.
\begin{enumerate}
\item Read the images("img1.jpg","img2.jpg")
\item Read the correspondences from the file "corrs.txt" 
\item Choose the number of iterations $N$ as 10000 initially.
\item Select 4 random correspondences.
\item Compute homography(H) with these correspondences by applying DLT algorithm that we have seen in the previous lab.
\item Transform points from reference image to query image for each potential correspondence between the images: $x' = H * x $
\item Count number of inliers as follows:\\
    --- Check whether there exists any point close to transformed points with the Euclidean distance at most \textbf{3 pixels}.
\item Compute the inlier ratio as follows:\\
\begin{equation}
\text{inlierRatio} = \% \frac{\text{numberOfInliers}}{\text{numberOfCorrespondences}}
\end{equation}
\item Update $N$ as follows:\\
\begin{enumerate}

\item Calculate the probability($w$) of each of the randomly selected correspondence is inlier.
\item $1 - ( 1 - w^{s})^{N} >= 0.99$. From there, calculate N. 

\end{enumerate}
\item Repeat step[4-9] N times and save the homography that gives the maximum inlier ratio.

\item Warp the second image with the inverse of the final computed homography.
\end{enumerate}
\end{document}